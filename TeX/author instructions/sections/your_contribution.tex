\section{Heuristic augmentation of SAT solvers}
In computer science typically the third section contains an exposition of the main ideas, for example the development of a theory, the analysis of the problem (some proofs), a new algorithm, and potentially some theoretical analysis of the properties of the algorithm.

Our research is a combination between improvements to an existing idea and experimental work. There already exist approaches to solving RCPSP with SAT solvers, and the research focuses on extending this approach with heuristic variable selection. This extension can only be evaluated by performing experimental work. %Some more detailed suggestions for typical types of contributions in computer science are described in the following subsections.


\subsection*{Experimental work}
In this case, this section will mostly contain a description of the methods/algorithms you will be comparing. Although not all methods need to be described in detail (providing appropriate references are available), make sure that you reveal sufficient details to a reader not familiar with these methods to: a) obtain a high-level understanding of the method and differences between them, and b) understand your explanation of the results/conclusions.

\subsection*{Heuristic}
This subsection should explain how the heuristic is calculated, and whether it provides and advantage compared to a standard approach to solving the specific sub-problem.

\subsection*{CNF Encoding}
This subsection should describe how the input data is encoded into CNF that will be solved by some SAT solver described in the next subsection. 
Previous approaches of encoding a RCPSP into CNF might be referenced and compared.

\subsection*{SAT solver}
This subsection should describe how the SAT solver finds optimal solutions.
Also explain why certain SAT solvers were selected for solving.
If possible cover the inclusion of heuristics for variable selection in the sat solver.