\documentclass[english]{article}
\usepackage[T1]{fontenc}
\usepackage[latin9]{inputenc}
\usepackage{geometry}
\geometry{verbose,tmargin=3cm,bmargin=3cm,lmargin=3cm,rmargin=3cm}

\makeatletter
\usepackage{url}

\makeatother

\usepackage{babel}


\title{Research Plan for Combinatorial Optimisation for Scheduling}
\author{Jasper Vermeulen}
\date{\today}

\newcommand{\namelistlabel}[1]{\mbox{#1}\hfil}
\newenvironment{namelist}[1]{%1
\begin{list}{}
    {
        \let\makelabel\namelistlabel
        \settowidth{\labelwidth}{#1}
        \setlength{\leftmargin}{1.1\labelwidth}
    }
  }{%1
\end{list}}

\begin{document}
\maketitle


\section*{Background of the research}
The problem of scheduling tasks arises in industries all the time. It is not hard to imagine that generating an optimized schedule can be of great profit for production or logistic operations.
Optimization can for example be minimizing the overall required time or minimizing the delay before starting a task. Because this type of problem is so prevalent it has already been subject to much research.\\
Formally this specific type of problem is known as the resource-constrained project scheduling problem (RCPSP). On its own the problem definition for RCPSP is to limited to be of use for realistic application. To make sure the researched algorithms solving the scheduling problem would have a wider use case many variations and extensions to the problem definition have been made over time \cite{RN6}.


\section*{Research Question}
Can the addition of a heuristic in a SAT solver algorithm for PRCPSP-ST models reduce the average time needed for each intermediate schedule it creates?
Can the addition of a heuristic for variable selection in the SAT solver algorithm applied to PRCPSP-ST models reduce the average of resulting schedule makespans?

\section*{Method}
TBD in the final version of the research plan.

\section*{Planning of the research project}
To make finish the project successfully a planning is outlined in this section. For each week the meetings and presentations are outlined in the fist subsection. The next subsection contains all the activities that should be performed and lastly a list of deliverables is provided to keep track of all parts required for the following weeks.
\subsection{Week 1}
\subsubsection{Meetings}

\begin{tabular}{ l c l }
  \hline			
  Participants & Objective & Date and time \\
  \hline\hline
  Peers + supervisor & Detailing research topic & \date{April 20 9:00 am} \\
  Peers & Discussing research plan and background information & \date{April 21} or \date{April 22} \\
  \hline  
\end{tabular}

\subsubsection{Activities}
\begin{tabular}{ l p{0.6\textwidth}}
  \hline			
  Activity & Objective \\
  \hline\hline
  Read reference paper from project forum & {\begin{itemize}
    \item Gain insight in the RCPSP problem
    \item Learn about the pre-emption variant of the problem
    \item Learn a way to model the problem
    \item Analyse the use of the model in a SAT-solver
\end{itemize}{}}  \\
  Analyse research topic & {\begin{itemize}
    \item Formulate the research topic
    \item Formulate a research question
    \item Derive sub-questions
    \item Make search queries
    \item Selecting information sources
    \item Store information sources
    \item Generate a literature list
\end{itemize}{}} \\
  Read abstracts from literature list & {\begin{itemize}
    \item Make tags for literature list
    \item Find an example for a RCPSP problem model
    \item Gather information for the background of the research
\end{itemize}{}} \\
  Transform sub-questions into tasks & {\begin{itemize}
    \item Find the required tasks to answer the research question
    \item Make a time-line for the remaining 9 weeks of the project
\end{itemize}{}} \\
  Make a list of tools/software/data & {\begin{itemize}
    \item Making sure all required parts are accessible
    \item Checking for completeness with supervisor
\end{itemize}{}} \\
  Write research plan & {\begin{itemize}
    \item Finishing the first deliverable to get feedback from supervisor
\end{itemize}{}} \\
  \hline  
\end{tabular}

\subsubsection{Deliverables}
After all the activities have been finished the following deliverables should have been made:
{\begin{itemize}
    \item Research question
    \item Sub-questions
    \item Information sources
    \item Literature list (with tags)
    \item Example for a RCPSP problem model
    \item Written background research
    \item List of tasks
    \item Time-line for tasks and official deadlines
    \item List of tools/software/data
    \item Final version research plan
\end{itemize}{}}


\bibliographystyle{plain}
\bibliography{ProposalBib}
\end{document}
