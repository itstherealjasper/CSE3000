\documentclass[english]{article}
\usepackage[T1]{fontenc}
\usepackage[latin9]{inputenc}
\usepackage{geometry}
\geometry{verbose,tmargin=3cm,bmargin=3cm,lmargin=3cm,rmargin=3cm}

\makeatletter
\usepackage{url}

\makeatother

\usepackage{babel}


\title{Research Plan for Combinatorial Optimisation for Scheduling}
\author{Jasper Vermeulen}
\date{\today}

\newcommand{\namelistlabel}[1]{\mbox{#1}\hfil}
\newenvironment{namelist}[1]{%1
\begin{list}{}
	{
		\let\makelabel\namelistlabel
		\settowidth{\labelwidth}{#1}
		\setlength{\leftmargin}{1.1\labelwidth}
		}
	}{%1
\end{list}}

\begin{document}
\maketitle

\section*{Background of the research}
The problem of scheduling tasks arises in industries all the time. It is not hard to imagine that generating an optimized schedule can be of great profit for production or logistic operations.
Optimization can for example be minimizing the overall required time or minimizing the delay before starting a task. Because this type of problem is so prevalent it has already been subject to much research.\\
Formally this specific type of problem is known as the resource-constrained project scheduling problem (RCPSP). On its own the problem definition for RCPSP is to limited to be of use for realistic application. To make sure the researched algorithms solving the scheduling problem would have a wider use case many variations and extensions to the problem definition have been classified over time \cite{RN9}, \cite{RN10}. More recently the variations and extensions have also been surveyed and put into a structured overview \cite{RN6}.\\
For this research the preemptive resource-constrained project scheduling problem with setup times (PRCPSP-ST) variant is under study. Preemption allows an activity to be interrupted during its scheduled time by another activity. Each interruption can be seen as a split of the activity into multiple smaller activities. The setup times are introduced for each interruption in an activity to discourage endless splits resulting in a chaotic schedule. This model has already been established \cite{RN13} and a proposed algorithm was found to result in a reduction of the makespan \cite{RN1} compared to the optimal schedule without activity preemption. Within this algorithm the activities are split into all possible integer time segments and a SAT solver makes a selection from these segments. The resulting list is used to construct a schedule with a genetic algorithm \cite{RN14}.\\

%The idea now is to include a heuristic for the variable selection in the dpll algorithm to improve the resulting schedule makespan


\section*{Research Question}
Can the addition of a heuristic for activity segment selection in the SAT solver algorithm applied to PRCPSP-ST models reduce the average makespan of the resulting schedules generated from the selected activity segment list in either an equal amount of iterations or the same amount time?
{\begin{itemize}
    \item What is a RCPSP activity-on-the-node network?
    \item How can a RCPSP activity-on-the-node network be transformed to include activity preemption?
    \item How can setup times be included in the activity network for splits in an activity schedule?
    \item What is the role of a SAT solver in an RCPSP optimization algorithm?
    \item What is the required input of a SAT solver?
    \item How can PRCPSP-ST problems be modelled as an input to a SAT solver for activity segment selection?
    \item How can the output of a SAT solver be used to construct an activity schedule?
    \item What could be a possible heuristic for a SAT solver?
    \item Where can a heuristic be used by a SAT solver?
    \item Why could a heuristic be beneficial for solving PRCPSP-ST problems?
    \item How does the heuristic change the resulting activity schedule?
    \item How does the heuristic change the time required by the SAT solver?
\end{itemize}{}}


\section*{Method}
TBD in the final version of the research plan.

\section*{Planning of the research project}
To make finish the project successfully a planning is outlined in this section. For each week the meetings and presentations are outlined in the fist subsection. The next subsection contains all the activities that should be performed and lastly a list of deliverables is provided to keep track of all parts required for the following weeks.
\subsection{Week 1}
\subsubsection{Meetings}

\begin{tabular}{ l c l }
  \hline			
  Participants & Objective & Date and time \\
  \hline\hline
  Peers + supervisor & Detailing research topic & \date{April 20 9:00 am} \\
  Peers & Discussing research plan and background information & \date{April 21} or \date{April 22} \\
  \hline  
\end{tabular}

\subsubsection{Activities}
\begin{tabular}{ l p{0.6\textwidth}}
  \hline			
  Activity & Objective \\
  \hline\hline
  Read reference paper from project forum & {\begin{itemize}
    \item Gain insight in the RCPSP problem
    \item Learn about the pre-emption variant of the problem
    \item Learn a way to model the problem
    \item Analyse the use of the model in a SAT-solver
\end{itemize}{}}  \\
  Analyse research topic & {\begin{itemize}
    \item Formulate the research topic
    \item Formulate a research question
    \item Derive sub-questions
    \item Make search queries
    \item Selecting information sources
    \item Store information sources
    \item Generate a literature list
\end{itemize}{}} \\
  Read abstracts from literature list & {\begin{itemize}
    \item Make tags for literature list
    \item Find an example for a RCPSP problem model
    \item Gather information for the background of the research
\end{itemize}{}} \\
  Transform sub-questions into tasks & {\begin{itemize}
    \item Find the required tasks to answer the research question
    \item Make a time-line for the remaining 9 weeks of the project
\end{itemize}{}} \\
  Make a list of tools/software/data & {\begin{itemize}
    \item Making sure all required parts are accessible
    \item Checking for completeness with supervisor
\end{itemize}{}} \\
  Write research plan & {\begin{itemize}
    \item Finishing the first deliverable to get feedback from supervisor
\end{itemize}{}} \\
  \hline  
\end{tabular}

\subsubsection{Deliverables}
After all the activities have been finished the following deliverables should have been made:
{\begin{itemize}
    \item Research question
    \item Sub-questions
    \item Information sources
    \item Literature list (with tags)
    \item Example for a RCPSP problem model
    \item Written background research
    \item List of tasks
    \item Time-line for tasks and official deadlines
    \item List of tools/software/data
    \item Final version research plan
\end{itemize}{}}


\bibliographystyle{plain}
\bibliography{ProposalBib}
\end{document}
