\section{Conclusions and Future Work}
RCPSP has been around for some time now and a lot of research has gone into optimizing algorithms that can solve it efficiently. For the large amount of practical application that require additional or custom constraints an incrasing number of variations of the problem have become the subject of study. In this paper an exact appraoch in the form of a satisfiability encoding for the preemption extension with setup times has been proposed and tested against a heuristic solver. When tested both methods were able to produce schedules that were close to optimal non-preemptive schedule or the best known non-preemptive schedules. After establishing both solutions are capable of producing result close to establised results a comparison between the two algorithms could be maded. Comparing the approaches showed benefit in the exact SAT solver method if makespan reduction is necassary enough to sacrifice intermediate solutions in case of a timeout. This was shown by the over 90\% of equal of improved instance solutions from the SAT solver over the heuristic solutions when the SAT solver found a solution in time for the J30 dataset.

Unfortunatly the results for the RG30 dataset are not close enough to the best known solutions. Especially the SAT solver struggled to come up with solutions and when it did the results were still unimpressive. A possible explanation is the high variance in how serial/parallel the network structure of the instances are. This could not be confirmed within time for this research though and could be a start for a follow up study. Because the heuristic approach also showed worse results on the RG30 dataset it can at least be concluded that some factor other than project size causes the problems to be harder when preemption is allowed.

Future research could focus on comparing different SAT solvers once the problems are encoded into satisfiability clauses. Solvers are often optimized for different objectives and some might find an intermediate solution quicker or go for the optimal solution directly. A second suggestion is converting the proposed SAT encoding into a more generalized satisfiability modulo theories encoding. This encoding has native support for formulas including integers and makes encoding the resource constraints more straitforward. Another great way to build on the work done in this paper is changing the Pumpkin MAX-SAT solver to be more specialized on the specific instance. The size of the CNF encodings are really big because of the extended network after each task has been expanded to all its possilbe subsegments. An example would be to provide the solver with an initial solution created by the first iteration of the heuristic algorithm as a starting point. At the very least an implementation in this way will always procude a result when the encoding can be made within time.

Allowing preemption for project tasks seems to be able to provide lower makespan schedules. If future research can find the optimal solutions within the time limit with higher consistency it seems like it will be well worth the effort.