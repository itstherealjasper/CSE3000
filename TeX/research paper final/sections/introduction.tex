\section{Introduction}
The problem of scheduling tasks arises in industries all the time. It is not hard to imagine that generating an optimised schedule can be of great profit for production or logistic operations. For example, optimisation can minimise the overall required time or minimise the delay before starting a task. Because this type of problem is so prevalent it has already been subject to much research.

Formally this specific type of problem is known as the resource-constrained project scheduling problem (RCPSP) \cite{RN69, RN66, RN70}. Problems of this type are considering the allocation of tasks to resources within a project. Tasks can be anything that needs to be done during a project from people on a piece of equipment to a simulation run by a computer. Resources are made available for a project and in case of the examples are human experts assigned to the project and time reserved on computers to run the simulations. Tasks are given a duration required to finish and a required amount of a resource during processing. Resources are limited in availability during the project, eg. only one expert is available at a single moment in time so multiple tasks requiring an expert must be processed after one another. An important addition to this setup is the order in which pairs of tasks must be processed. Some tasks require another task to be finished before it can be started. This requirement is referred to as a precedence constraint.

It can be required by an application of an algorithm to include additional constraints on schedules. To provide for these additional constraints variations and extensions to the problem definition have been classified over time \cite{RN9}, \cite{RN10}. More recently, the variations and extensions have also been surveyed and put into a structured overview \cite{RN6}.

%mention the importance of your variant in specific. For example, why does adding preemption increase “realism”? Where can it be used? 

For this research, the preemptive resource-constrained project scheduling problem with setup times (PRCPSP-ST) variant is under study. Preemption allows a task to be interrupted during its scheduled time by another task. Each interruption can be seen as splitting the task into multiple smaller activities. The setup times are introduced for each interruption in a task to discourage endless splits resulting in a chaotic schedule. Both a model for allowing preemption \cite{RN21} and including setup times \cite{RN13} have already been established. Both models have been combined and a proposed algorithm for it was found to result in a reduction of the makespan compared to the optimal schedule without task preemption \cite{RN1}. Within this algorithm, the activities are split into all possible integer time segments and a satisfiability (SAT) solver selects a feasible subset from these segments \cite{RN3}. The resulting list was used to construct a schedule with a genetic algorithm established in earlier research \cite{RN14}.

In the research done on solving RCPSP variants, the focus has been on heuristic and meta-heuristic algorithms. These algorithms are usually variants of branch-and-bound algorithm \cite{RN21} or a form of genetic algorithms \cite{RN28} that were established for the standard RCPSP.

SAT solvers are a general tool that can be used on any algorithmic problem if it is encoded as the required input for the solver. A SAT solver is a program that tries to solve Boolean satisfiability problems. Boolean satisfiability problems are a set of true or false variables that are set up in a formula with ‘AND’ and ‘OR’ operators (sets of variables and operators are often referred to as clauses). The SAT solver tries to assign values to all variables in such a way that the outcome of the overall formula becomes true (also referred to as being satisfied). If a Boolean formula encoding can be found for a problem, a SAT solver can be used to try and solve it. In addition to finding a solution to the problem, it can include information about the solution it provides like the assignment for the variables it found or a minimal set of unsatisfiable clauses.

SAT solvers have been used as a part of the algorithm but using a SAT solver as a complete solution to try and solve PRCPSP-ST instances has not been researched before. Because RCPSP is known to be strongly NP-hard \cite{RN20}, a SAT solver might not be efficient enough to outperform the heuristic and meta-heuristic methods. But what a SAT solver does provide is a way to prove if a found solution is optimal and can therefore not be improved further. When looking for a reduction in makespan getting confirmation on optimality might be worth some possible trade-offs a SAT solver introduces.

Because there is room to try and find out if a SAT solver can match or even outperform the heuristic and meta-heuristic algorithms in the resulting schedules it produces the main research question is \textit{Can a satisfiability encoding of PRCPSP-ST models solved on a SAT solver be used reduce the average makespan of the resulting schedule compared to a heuristic algorithm when run for an equal amount time?}. The expectation for this research is to first make a heuristic algorithm that can solve the extended RCPSP instances with preemption and setup times and produce baseline results. And next, show that a satisfiability encoding run on a SAT solver can match or improve these results as well as show in how many cases it returns the optimal solution. These results can be used in future applications and research when having the minimum makespan is important enough to have the algorithm also provide proof of optimality.


*Report section structure goes here*