\section{Heuristics and augmentation of SAT solvers}
To evaluate the performance of a heuristically augmented SAT solver on PRCPSP-ST instances comparative data needs to be generated.  To make sure the variables between algorithms are limited they were all designed, implemented and run on the same hardware for this research. This way the at least coding skills and hardware improvements over time are not introduced in the data. Improved performance of new algorithms only as a result of the use of more recent hardware has shown up in the past as described in section \ref{responsible research}. Three algorithms are used to solve the same problem instances for an equal amount of time: a heuristic algorithm, SAT solver, heuristically augmented SAT solver.

The heuristic algorithm is an adapted version of the iterated greedy algorithm \cite{RN32}. It was designed for flow-shop scheduling but with a few tweaks it can also be applied to RCPSP.

For a SAT solver to be used the PRCPSP-ST has to be encoded into conjunctive normal form boolean logic. When the encoding is made any SAT solver is able to provide feasible schedules. Because there is a clear objective to reduce the makespan a more advanced MAX-SAT solver is used. This will create feasible schedules while also trying to optimize to an objective function.

To augment a SAT solver for a specific problem the code of the solver has to be changed. In the case of this research a heuristic function will be added that improves the selection of a possible variable. The heuristic function will use knowledge about PRCPSP-ST to try and select a variable that will reduce the resulting schedule makespan.

\subsection{Heuristic}
As a heuristic solution a tweaked version of the iterated greedy algorithm is implemented \cite{RN32}. This algorithm requires an activity list representation of the project. It start with a setup of an initial schedule and then iterates over a destruction phase and a construction phase until a time limit or number of iterations limit is reached. 

An activity list representation allows a serial generation scheme to construct a feasible schedule \cite{RN46}. The activity list represent a project as permutation vector of all the tasks. It is required that no task appears in the list after any of its successors. The serial generation scheme schedules all the task in the order of the list at the earliest possible start time that does not break precedence or resource restrictions. Because the generation scheme uses the tasks in order of the list tasks close to the front of the list can be seen as having a higher priority and are scheduled sooner by the algorithm. When the algorithm has scheduled all tasks the result is a left-justified schedule.

The initial schedule is generated with the use of a greedy heuristic. Firstly a resource utility rate \(u_i\) is calculated for each task 
\begin{equation}
u_i=\frac{d_i \times r_{i,k}}{a_k} 
\end{equation}
For each task its resource requirement is divided by the resource availability. The result is multiplied by the task duration. Next all tasks are put into a list and ordered by non-increasing resource utility rate. After ordering each task is moved directly in front of the first successor in the list. The result is an activity list representation and the serial generation scheme is run on it to create the initial (left-justified) schedule.

After the initial list is generated the main iterative part of the algorithm starts with the destruction phase. During this phase a copy is made of the initial schedule and next \(d=\lceil\frac{|N|}{4}\rceil\)  tasks are removed from the activity list at random. These are picked one by one and are kept separately in the order they were removed.

The second step of the main iteration is the construction phase. From the removed tasks the first is picked and placed at any index in the remaining activity list that doesn't break the precedence order. For each possible index the makespan of the left-justified schedule made with the serial generation scheme is calculated. The index with the lowest makespan is chosen and the task is inserted at the index. This process is repeated for each removed task until all tasks are in the activity list. At this point the resulting schedule makespan is compared to the initial schedule makespan and when an improved makespan has been found the initial schedule is overwritten by the new schedule.

This heuristic solution can any number of iterations of the destruction and construction phases until either a iterations limit is reached or a time limit is reached. At that point it will return the most optimal schedule it has found.


\subsection{CNF Encoding}
The CNF encoding is very much still in process and is therefore not a part of the draft version.
%Variable \(w_{i,x,y,t}\) is equal to 1 if activity segment \(S^i_xD_y\) starts at time \(t\), and 0 otherwise.
%\begin{align}
%\sum_{y=1}^{d_i} \sum_{x=max(c-y+1,1)}^{min(d_i-y+1,c)} \sum_{t=0}^{T-y} w_{i,x,y,t}&=1	&	\all i\in N; c=1,...,d_i
%\end{align}

%\begin{equation}
%\neg w_{i,x,y} \bigvee_{t\in{0,...,T-y}} s_{i,x,y,t}
%\end{equation}

\subsection{SAT solver}
Due to difficulties during the CNF encoding my current planning does not allow for any time spent on improving the SAT solver.