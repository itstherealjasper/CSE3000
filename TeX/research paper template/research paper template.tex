\documentclass[english]{article}
\usepackage[T1]{fontenc}
\usepackage[latin9]{inputenc}
\usepackage{geometry}
\geometry{verbose,tmargin=3.5cm,bmargin=4cm,lmargin=3.8cm,rmargin=3.8cm}

\makeatletter
\usepackage{url}
\usepackage{lipsum}  

\makeatother

\usepackage{babel}
\begin{document}

\title{Research Paper Template}

\author{YOUR NAME\footnote{\texttt{\{YOU\}@student.tudelft.nl}}\\
Supervisor(s): SUPERVISOR1, SUPERVISOR2\footnote{\texttt{\{SUPERVISOR1, SUPERVISOR2\}@tudelft.nl}}\\
EEMCS, Delft University of Technology, The Netherlands
}

%\date{}

\maketitle

\begin{abstract}
    Scheduling has been subject to much research. The resource-constrained project scheduling problem (RCPSP) is no exception. With the multiple different variations and additions to the standard definition that are possible, many exact, heuristic and meta-heuristic approaches have been proposed. One of those variations is allowing tasks in the project to be split up into smaller subsegments and scheduled at different times. The splitting of tasks is introduced to try and find project schedules that require less overall time to complete.

    This paper proposes a satisfiability (SAT) encoding for the RCPSP while also allowing tasks to be split. A SAT solver is used to solve the encoded problem instances and compared with the results of a heuristic algorithm. The comparison shows a lower rate of solved instances but found results are in almost all cases more optimal and proof of optimality can be provided. For applications where the sacrifice of completing tasks without splitting is only worthwhile if the absolute lowest schedule duration can be found this method provides a good alternative to other approaches.
\end{abstract}

\section{Introduction}
The problem of scheduling tasks arises in industries all the time. It is not hard to imagine that generating an optimised schedule can be of great profit for production or logistic operations. For example, optimisation can minimise the overal required time or minimise the delay before starting a task. Because this type of problem is so prevalent it has already been subject to much research.

Formally this specific type of problem is known as the resource-constrained project scheduling problem (RCPSP). The standard RCPSP has a set of tasks that each require a specified resource for a given amount of time and precedence restrictions are given for the tasks. On its own, this problem definition for standard RCPSP is limited and of little use to realistic applications where additional constraints must be followed or more options are available.
To make sure the researched algorithms solving the scheduling problem would have a wider use case, many variations and extensions to the problem definition have been classified over time \cite{RN9}, \cite{RN10}. More recently, the variations and extensions have also been surveyed and put into a structured overview \cite{RN6}.

%mention the importance of your variant in specific. For example, why does adding preemption increase “realism”? Where can it be used? 

For this research, the preemptive resource-constrained project scheduling problem with setup times (PRCPSP-ST) variant is under study. Preemption allows an activity to be interrupted during its scheduled time by another activity. Each interruption can be seen as splitting the activity into multiple smaller activities. The setup times are introduced for each interruption in an activity to discourage endless splits resulting in a chaotic schedule. Both a model for allowing preemption \cite{RN21} and including setup times \cite{RN13} have already been established. Both models have been combined and a proposed algorithm for it was found to result in a reduction of the makespan compared to the optimal schedule without activity preemption \cite{RN1}. Within this algorithm, the activities are split into all possible integer time segments and a SAT solver makes a selection from these segments \cite{RN3}. The resulting list was used to construct a schedule with a genetic algorithm established in earlier research \cite{RN14}.

In the research done on solving RCPSP variants, the focus has been on heuristic and meta-heuristic algorithms. These algorithms are usually variants of branch-and-bound algorithm \cite{RN21} or a form of genetic algorithms \cite{RN28} that were established for the standard RCPSP.

The use of SAT solvers as a part of the algorithm was mentioned earlier but using a SAT solver as a complete solution to try and solve PRCPSP-ST instances has not been researched before. A SAT solver is a very general tool that can be used on any algorithmic problem as long as it is encoded as the required input for the solver. Heuristic algorithms are specialised by knowledge or an insight into the problem variation translated into a heuristic rule for the algorithm to use. Because the RCPSP is known to be strongly NP-hard \cite{RN20}, the SAT solver might not be more efficient than the heuristic and meta-heuristic at first. But, a SAT solver can also be modified to include a heuristic for the PRCPSP-ST variant. In a way, the addition of a heuristic can be seen as an augmentation of the SAT solver. The augmentation is making the solver more specialised for the PRCPSP-ST variant and could result in a better performance than other algorithms. 

Because there is room to try and find out if a SAT solver can outperform the heuristic and meta-heuristic algorithms the main research question is: \textit{Can the addition of a simple heuristic to a SAT solver algorithm used to solve to PRCPSPST models reduce the average makespan of the resulting schedule when run for an equal amount time?}. The expectation for this research is to first show that a SAT solver can be used to solve PRCPSP-ST instances. And next, show that the heuristically augmented version of the SAT solver algorithm will result in a lower makespan than a heuristic algorithm when running an equal amount of time.

*Report section structure goes here*

\section{Methodology or Problem Description}
Choose one that fits your research best:
\subsection{Methodology}
Typically in general research articles, the second section contains a description of the research methodology, explaining what you, the researcher, is doing to answer the research question(s), and why you have chosen this method.
For purely analytical work this is a description of the data collection or experimental setup on how to test the hypothesis, with a motivation.
In any case this section includes references to necessary background information.
For a survey paper this includes the method of how you arrived at the set of papers included in the survey.

\subsection{Formal Problem Description}
For some types of work in computer science the methodology is standard: analyze the problem (e.g., make assumptions and derive properties), present a new algorithm and its theoretical background, proving its correctness, and evaluate unproven aspects in simulation.
Then an explanation of the methodology is often omitted, and the setup of the evaluation is part of a later section on the evaluation of the ideas.\footnote{This already shows that there is no single outline to be given for all papers.}
In this case, explain relevant (background) concepts, theory and models in this section (with references) and relate them to your research question.
Also this section then typically contains a more precise, formal description of the problem.

Do not forget to give this section another name, for example after the problem you are solving.


\section{Heuristic and SAT solver}
Two algorithms were implemented for this research to be compared: a heuristic algorithm and a boolean satisfiability encoding run on a SAT solver.

The heuristic algorithm is an adapted version of the iterated greedy algorithm \cite{RN32}. It was designed for flow-shop scheduling but with a few tweaks it can also be applied to RCPSP.

For a SAT solver to be used the PRCPSP-ST has to be encoded into conjunctive normal form boolean logic. When the encoding is made any SAT solver is able to provide feasible schedules. Because there is a clear objective to reduce the makespan a more advanced MAX-SAT solver is used. A MAX-SAT solver expand the features of a SAT solver by adding native support for an objective function. This objective function gives a score to each solution found and when running a MAX-SAT solver it keeps trying to maximize this score until no further improvements can be found.

\subsection{Heuristic}
As a heuristic solution a tweaked version of the iterated greedy algorithm is implemented \cite{RN32}. This algorithm requires an activity list representation of the project. It start with a setup of an initial schedule and then iterates over a destruction phase and a construction phase until a time limit or number of iterations limit is reached. 

An activity list representation allows a serial generation scheme to construct a feasible schedule \cite{RN46}. The activity list represent a project as permutation vector of all the tasks. It is required that no task appears in the list after any of its successors. The serial generation scheme schedules all the task in the order of the list at the earliest possible start time that does not break precedence or resource restrictions. Because the generation scheme uses the tasks in order of the list tasks close to the front of the list can be seen as having a higher priority and are scheduled sooner by the algorithm. When the algorithm has scheduled all tasks the result is a left-justified schedule.

The initial schedule is generated with the use of a greedy heuristic. Firstly a resource utility rate \(u_i\) is calculated for each task 
\begin{equation}
u_i=\frac{d_i \times r_{i,k}}{a_k} 
\end{equation}
For each task its resource requirement is divided by the resource availability. The result is multiplied by the task duration. Next all tasks are put into a list and ordered by non-increasing resource utility rate. After ordering each task is moved directly in front of the first successor in the list. The result is an activity list representation and the serial generation scheme is run on it to create the initial (left-justified) schedule.

After the initial list is generated the main iterative part of the algorithm starts with the destruction phase. During this phase a copy is made of the initial schedule and next \(d=\lceil\frac{|N|}{4}\rceil\)  tasks are removed from the activity list at random. These are picked one by one and are kept separately in the order they were removed.

The second step of the main iteration is the construction phase. From the removed tasks the first is picked and placed at any index in the remaining activity list that doesn't break the precedence order. For each possible index the makespan of the left-justified schedule made with the serial generation scheme is calculated. The index with the lowest makespan is chosen and the task is inserted at the index. This process is repeated for each removed task until all tasks are in the activity list. At this point the resulting schedule makespan is compared to the initial schedule makespan and when an improved makespan has been found the initial schedule is overwritten by the new schedule.

This heuristic solution can any number of iterations of the destruction and construction phases until either a iterations limit is reached or a time limit is reached. At that point it will return the most optimal schedule it has found.


\subsection{CNF Encoding}
The conjunctive normal form encoding used for this research is based on existing work used to solve the RCPSP with SAT \cite{RN17}. This encoding was altered to include preemption.

For the new encoding a project has to be extended into a new network \(N^*\) by replacing each task with a new network of tasks that represents all possible ways the task could be preempted. This extension has already been documented but a short summary will be given \cite{RN1}. A task will be split into a set of all possible integer segments. From this new set of segments all chains of segments are generated that represent the original task in its entirety. These chains can now replace the original task in the task network. All segments are added as new tasks with precedence relations representing the segment chains. All predecessors of the task get an additional successor for each segment that contains the first integer part of the original task. Each of the segments that contain the last integer part of the original task must also have the original successors of the task.

With the extended network the earliest \(es_i\) and its latest \(ls_i\) start times, its ealiest \(ef_i\) and its latest \(lf_i\) finish times are calculated using the critical-part method by the Floyd-Warshall algorithm \cite{RN53}. With these values two boolean variables will be defined and used in the SAT clauses. For each task \(i \in N\) and \(t \in \{es_i,...,ls_i\}\) there is a start variable \(s_{i,t}\) which is true if activity \(i\) start at time \(t\) and for east task \(i \in N\) and \(t \in \{es_i,...,lf_i\}\) a process variable \(u_{i,t}\) which is true if activity \(i\) is in process at time \(t\).

Now the complete encoding can be made and it includes five types of clauses. The completion, consistency, precedence, resource and objective clauses. The first four are defined as hard clauses meaning the SAT solver must satisfy them. For the objective a set of soft clauses is used. The SAT solver will try and maximize the amount of soft clauses it can satisfy.

Completion clauses make sure that each task segment is processed once and therefore making sure that all the work in the project is done. New subsets are required to define the completion clauses. \(C^*_{i,l} \subseteq N^*\) each have as elements all task segments that contain time segment \(l\) of task \(i\). Equation \ref{eq:equation2} gives the mathematical definition of the completion clauses.
\begin{align}\label{eq:equation2}
\bigvee_{t\in {es_i,...,ls_i}} s_{it}   &&  j\in N; l\in \{0,...,d_j\}; i\in C^*_{j,l}
\end{align}

When a start variable of a task is set to true the consistency clauses given in equation \ref{eq:equation3} ensure that the required process variables of the task are also set to true.
\begin{align}\label{eq:equation3}
\neg s_{i,t} \vee u_{i,l}   &&  i\in N^*;t\in \{es_i,...,ls_i\};l\in \{t,...,t+d_i-1\}
\end{align}

A set of precedence clauses is introduced to satisfy the required precedence constraints. This is done by only allowing a task to have a start variable set to true if all predecessors started early enough to be finished by that time. This clause is given in equation \ref{eq:equation4}.
\begin{align}\label{eq:equation4}
\neg s_{i,t} \bigvee_{l=es_j,...,es_i-d_j} s_{j,l}  &&  (j,i)\in A; t\in\{es_i,...,ls_i\}
\end{align}

The resource clauses are defined as pseudo-boolean function that are converted into true CNF. The conversion is done by first building binary decision diagrams from the pseudo-boolean function. Next the binary decision diagrams are converted to a set of CNF clauses that represent the same pseudo-boolean function. The process of converting pseudo-boolean into SAT is known and researched \cite{RN38}. The used pseudo-boolean function is given in equation \ref{eq:equation5}.
\begin{align}\label{eq:equation5}
\sum^n_{i=1} u_{i,t}r_{i,k} \leq a_k    &&  t\in \{1,...,T\}; k\in R
\end{align}

Lastly there are the objective clauses in equation \ref{eq:equation6}. These are soft clauses and the SAT solver tries to satisfy as many of these clauses as possible.
\begin{align}\label{eq:equation6}
s_{i,t} &&  t\in \{1,...,T\}; i = \text{dummy finish task}
\end{align}

When this encoding is done the result is run on the Pumpkin MAX-SAT solver that was provided to the research by supervisor Emir Demirović.

% \subsection{SAT solver}
% Due to difficulties during the CNF encoding my current planning does not allow for any time spent on improving the SAT solver.

%\bigskip
%\lipsum[1-67]

\section{Experimental Setup and Results}
In order to test the performance of the different algorithmic approaches to solving PRCPSP-ST a number of experiments have been carried out.
Each experiment is run on the high performance computing cluster at the Delft University of Technology. The algorithms have access to 8GB RAM and 1 core of the Intel(R) Xeon(R) Gold 6248R CPU running at a base frequency of 3.00 GHz and max turbo frequency of 4.00 GHz.

\subsection{Project data}
To test the difference between the algorithmic approaches to solve instances with activity preemption a number of tests are performed using three different datasets. The complete datasets contain a different amount of problem instances (\# inst) and project ranging from 10 to 50 tasks (\# tasks). The J30 and RG30 datasets contain projects with 30 tasks and have 480 and 1800 instances respectively and the DC1 dataset has project ranging from 10 to 50 tasks also containing 1800 instances. For the experiments the a subset of the first 480 instances have been taken of all three datasets. This reduces the size of the projects in the DC1 dataset to a range from 10 to 20 tasks. The information about the datasets is summarized in table \ref{table:table1}.

The networks structures of the datasets has been subject to earlier research. One thing of note is the serial/parallel indicator \(I_2\) (later renamed to SP) of the datasets \cite{RN65,RN63}. This indicator measures how close the network structure of a project is to a complete serial of paralles network. The indicator shows that both the DC1 and J30 have a more limited SP indicator range and the RG30 has the largest range. This means that the RG30 dataset has the biggest variance in network structures for its project instances.

\begin{table}
	\begin{center}
		\caption{Summary of datasets used in the experiments}
		\label{table:table1}
		\begin{tabular}{ c | c c c c }
			Name & \# inst & \# tasks & subset size & \# tasks in subset \\
			\hline
			DC1 & 1800 & 10 - 50 & 480 & 10 - 20 \\
			J30 & 480 & 30 & 480 & 30 \\
			RG30 & 1800 & 30 & 480 & 30
		\end{tabular}
	\end{center}
\end{table}

The setup time penalty \(s\) is set to 1, 2 and 5 time units to test the impact on the overall makespan. These values are chosen to be around .1, .2 and .5 times the length of the longer tasks in the datasets that are around 10 time units.

To solve the instances the tweaked version of the iterated greedy heuristic is used to calculate a baseline and the CNF encoding run on the Pumpkin MAX-SAT solver is used to calculate data to compare to the baseline. Each algorithm is run for 60 seconds of CPU time on each instance.

\subsection{Performance indicators}
The percentage of schedules that can be reduced below the known optimal solution by allowing preemption is calculated to motivate why introducing preemption can be beneficial for certain projects. This value is the calculated by taking the percentage of instances within a dataset for a certain setup time that is lower than the known optimal makespan. Additionally the deviation percentage of the lower makespan can give an idication of the amount of time saved and is also calculated.

To assess the performance two test values are calculated. First the average percentage deviation from the either the known optimal makespan or the lower bound makespan without activity preemption is calculated. This is used to check if both algorithms can come up with schedules that are reasonably close to existing solutions. The second test value is the percentage of makespans that is improved by the CNF endocing run on the Pumpkin MAX-SAT solver compared to the iterated greedy heuristic solution. This measure will show if using a CNF encoding instead of a heuristic approach without any further optimization could be used to find more optimal schedules.

During the assessment of the two test values it has to be taken into account that the CNF encoding solved by a SAT solver might not produce a single schedule in the given time limit. So a percentage of instances where the SAT solver finds a solution is also calculated. This number will indicate the tradeoff for finding a proven optimal solution when time to create a schedule is limited.

\subsection{Results}
The experiments described have been performed and the data is collected and summarized.

Table \ref{table:table2} shows the percentage of instances for which the algorithms could find a lower makespan when preemption is allowed (\%Imp). For the lower setup times of 1 and 2 time units the heuristic approach could find reduces makespans in 0.8 to 13\% of instances. This number drops down to almost 0\% for the high setup time of 5 time units. The SAT solver approach shows a similar range of 1.9 to 13.6\% for the lower setup times but this number stays more consistent at a range of 1.5 to 10.8\% when the setup times are increased to 5 time units.

\begin{table}
	\begin{center}
		\caption{Heuristic and SAT algorithm percentage of makespans reduced by allowing preemption}
		\label{table:table2}
		\begin{tabular}{ c | c c c }
			Dataset & \(s\) & \%Imp by heuristic & \%Imp by SAT \\
			\hline
			DC1  & 1 & 13 \% & 12 \% \\ 
			  & 2 & 4.5 \% & 14 \% \\  
			  & 5 & 0.34 \% & 11 \% \\ 
			J30  & 1 & 3.5 \% & 5.8 \% \\ 
			  & 2 & 1.4 \% & 6.7 \% \\  
			  & 5 & 0 \% & 4.3 \% \\ 
			RG30 & 1 & 0.84 \% & 1.9 \% \\ 
			 & 2 & 0.63 \% & 1.5 \% \\  
			 & 5 & 0 \% & 1.5 \%
		\end{tabular}
	\end{center}
\end{table}

For the improved instances the deviation percentage (\%Dev) from the known optimal solutions are shown in table \ref{table:table3}. When the heuristic algorithm finds an improved schedule the deviation percentage from the known optimal solution is around -3.8\% for the DC1 dataset, -1.2\% for the RG30 dataset and -2.5\% for the J30 dataset. The SAT solver algorithm has values of -4.4\%, -3.4\% and -1.6\% for those datasets respectively. These are the averaged values over the different setup times.

\begin{table}
	\begin{center}
		\caption{Heuristic and SAT algorithm \%Dev of optimal makespan for improved instances}
		\label{table:table3}
		\begin{tabular}{ c | c c c }
			Dataset & \(s\) & \%Dev heuristic & \%Dev SAT \\
			\hline
			DC1  & 1 & -3.6 \% & -4.8 \% \\ 
			  & 2 & -3.4 \% & -4.2 \% \\  
			  & 5 & -4.5 \% & -4.1 \% \\ 
			J30  & 1 & -2.9 \% & -3.4 \% \\ 
			  & 2 & -2.1 \% & -3.7 \% \\  
			  & 5 & - & -3.2 \% \\ 
			RG30 & 1 & -1.2 \% & -3.0 \% \\ 
			 & 2 & -1.2 \% & -3.5 \% \\  
			 & 5 & - & -2.6 \%
		\end{tabular}
	\end{center}
\end{table}

Table 4 shows the average deviation percentage (\%Dev) of all instances compared to the known optimal solutions. For the DC1 dataset all deviations for both algorithms are close to 0. The results for dataset J30 shows a higher deviation of around 2 to 3\% for the heuristic algorithm and around 0.5 to 1.5\% for the SAT solve algorithm. The higher number indicates the algorithms could not find the known optimal solutions on average. For the RG30 dataset these numbers go up the highest at 5\% for the heuristic and 10\% for the SAT solve approach.

\begin{table}
	\begin{center}
		\caption{Deviations from known optimal or lower bound makespan solutions}
		\label{table:table4}
		\begin{tabular}{ c | c c c }
			Dataset & \(s\) & \%Dev heuristic & \%Dev SAT \\
			\hline
			DC1  & 1 & -0.16 \% & -0.18 \% \\ 
			  & 2 & -0.16 \% & 0.08 \% \\  
			  & 5 & 0.71 \% & -0.20 \% \\ 
			J30  & 1 & 1.9 \% & 1.4 \% \\ 
			  & 2 & 2.4 \% & 0.32 \% \\  
			  & 5 & 2.9 \% & 1.83 \% \\ 
			RG30 & 1 & 5.3 \% & 12 \% \\ 
			 & 2 & 5.8 \% & 10 \% \\  
			 & 5 & 5.6 \% & 9.5 \%
		\end{tabular}
	\end{center}
\end{table}

In table \ref{table:table5} the comparison between the results of the SAT solver compared to the heuristic algorithm is shown. For the DC1 and J30 datasets the number of instances for which the heuristic solution could find a solution strictly better than the SAT solver is aroung 9\% on average over the different setup times. The RG30 dataset shows that heuristic approach found lower makespans in around 70\% of the instances.These results are calculated for the instances where the SAT solver could find a solution.

\begin{table}
	\begin{center}
		\caption{Comparison of SAT solver solutions against heuristic algorithm}
		\label{table:table5}
		\begin{tabular}{ c | c c c }
			Dataset & \(s\) & \%Imp by SAT & \%Equal \\
			\hline
			DC1 & 1 & 13 \% & 80 \% \\ 
			  	& 2 & 20 \% & 60 \%\\  
			  	& 5 & 19 \% & 78 \%\\ 
			J30 & 1 & 30 \% & 58 \%\\ 
			  	& 2 & 39 \% & 56 \%\\  
			  	& 5 & 38 \% & 55 \%\\ 
			RG30 & 1 & 18 \% & 7 \%\\ 
			 	 & 2 & 24 \% & 6 \%\\  
			 	 & 5 & 27 \% & 8 \%
		\end{tabular}
	\end{center}
\end{table}

Finally an overview of the SAT algorithm performance is given in table \ref{table:table5}. This shows that the percentage of instances for which a schedules could be found (\%Satisfied) are 75.2\%, 28.8\% and 24.0\% in the DC1, J30 and RG30 datasets respectively. For found schedules the SAT solver also provides if it is proven to be optimal. The percentage of proven optimal solutions (\%Optimality proven) are 67.5\%, 81.9\% and 9.0\% for the DC1, J30 and RG30 datasets respectively.

\begin{table}
	\begin{center}
		\caption{SAT algorithm performance}
		\label{table:table6}
		\begin{tabular}{ c | c c c }
			Dataset & \%Satisfied & \%Optimality proven \\
			\hline
			DC1 & 75.2 & 67.5 \\
			J30 & 28.8 & 81.9 \\ 
			RG30 & 24.0 & 9.0
		\end{tabular}
	\end{center}
\end{table}



% \subsection{Setup time}
% This subsection is to show the impact of the setup times on the resulting makespans when allowing preemption. The experiments must still be run so this section remains empty for now.

\section{Responsible Research}
Reflect on the ethical aspects of your research and discuss the reproducibility of your methods.
Note that although in many published works there is no such a section (it may be part of some meta-information collected by the journal, or part of the discussion section), we require you to think (and report) about this as part of this course.

Main topic is the correct way to compare algorithm results. They should all be written by programmers of equal skill and be run on the same system.

Also discuss the importance of making source code publicly available, to make it easier for your work to be reviewed and to improve reproducibility. 


\section{Discussion}
This section can only be written correctly after all experiments have been run and all results are evaluated.

\section{Conclusions and Future Work}
RCPSP has been around for some time now and a lot of research has gone into optimizing algorithms that can solve it efficiently. For the large amount of practical application that require additional or custom constraints an incrasing number of variations of the problem have become the subject of study. In this paper an exact appraoch in the form of a satisfiability encoding for the preemption extension with setup times has been proposed and tested against a heuristic solver. When tested both methods were able to produce schedules that were close to optimal non-preemptive schedule or the best known non-preemptive schedules. After establishing both solutions are capable of producing result close to establised results a comparison between the two algorithms could be maded. Comparing the approaches showed benefit in the exact SAT solver method if makespan reduction is necassary enough to sacrifice intermediate solutions in case of a timeout. This was shown by the over 90\% of equal of improved instance solutions from the SAT solver over the heuristic solutions when the SAT solver found a solution in time for the J30 dataset.

Unfortunatly the results for the RG30 dataset are not close enough to the best known solutions. Especially the SAT solver struggled to come up with solutions and when it did the results were still unimpressive. A possible explanation is the high variance in how serial/parallel the network structure of the instances are. This could not be confirmed within time for this research though and could be a start for a follow up study. Because the heuristic approach also showed worse results on the RG30 dataset it can at least be concluded that some factor other than project size causes the problems to be harder when preemption is allowed.

Future research could focus on comparing different SAT solvers once the problems are encoded into satisfiability clauses. Solvers are often optimized for different objectives and some might find an intermediate solution quicker or go for the optimal solution directly. A second suggestion is converting the proposed SAT encoding into a more generalized satisfiability modulo theories encoding. This encoding has native support for formulas including integers and makes encoding the resource constraints more straitforward. Another great way to build on the work done in this paper is changing the Pumpkin MAX-SAT solver to be more specialized on the specific instance. The size of the CNF encodings are really big because of the extended network after each task has been expanded to all its possilbe subsegments. An example would be to provide the solver with an initial solution created by the first iteration of the heuristic algorithm as a starting point. At the very least an implementation in this way will always procude a result when the encoding can be made within time.

Allowing preemption for project tasks seems to be able to provide lower makespan schedules. If future research can find the optimal solutions within the time limit with higher consistency it seems like it will be well worth the effort.

\appendix
\section{Some further guidelines that go without saying (right?)}

\begin{itemize}
\item Read the manual for the Research Project. (See e.g.\ the instructions on the maximum length: less is more!)
\end{itemize}

\subsection{Reference use}
\begin{itemize}
\item use a system for generating the bibliographic information automatically from your database, e.g., use BibTex and/or Mendeley, EndNote, Papers, or \ldots
\item all ideas, fragments, figures and data that have been quoted from other work have correct references
\item literal quotations have quotation marks and page numbers
\item paraphrases are not too close to the original
\item the references and bibliography meet the requirements
\item every reference in the text corresponds to an item in the bibliography and vice versa
\end{itemize}

\subsection{Structure}
Paragraphs
\begin{itemize}
\item are well-constructed
\item are not too long: each paragraph discusses one topic
\item start with clear topic sentences
\item are divided into a clear paragraph structure
\item there is a clear line of argumentation from research question to conclusions
\item scientific literature is reviewed critically
\end{itemize}

\subsection{Style}
\begin{itemize}
\item correct use of English: understandable, no spelling errors, acceptable grammar, no lexical mistakes 
\item the style used is objective
\item clarity: sentences are not too complicated (not too long), there is no ambiguity
\item attractiveness: sentence length is varied, active voice and passive voice are mixed
\end{itemize}

\subsection{Tables and figures}
\begin{itemize}
\item all have a number and a caption
\item all are referred to at least once in the text
\item if copied, they contain a reference
\item can be interpreted on their own (e.g. by means of a legend)
\end{itemize}


\bibliographystyle{plain}
\bibliography{references}

A rule of thumb for dealing with the literature is the following: scan about 10--20 contributions: read title, abstract, part of introduction and conclusions; categorize contribution; some of these are studied in more depth: completely read about 5 conference papers or equivalent (summarize contribution in own words); of which studied in-depth about 2 conference papers (the student is able to explain in detail and criticize contributions). This may result in 5--20 references, possibly even more if the project is a literature study.

\end{document}
